\section{Introduction} \label{sec:introduction}

Financial markets are known to aggregate information by turning private views into public prices,
incorporating underlying private beliefs about the odds of potential future states of the world.
Prediction markets make this explicit: in them, contracts are traded whose payoffs are tied to the outcome of distinct events. When the payoff is fixed to \$1 if state of the world realises, and \$0 otherwise, intuition dictates the trading price be read like a probability.
While this seems appealing, it raises the question of what information these prices reflect. In other words, if they are read as probabilities, what probabilities are they measuring? This motivates the research topic of this thesis, the extent of price discovery in prediction markets.

This question could be explored in a multitude of ways. Since September 2024, prediction markets have enjoyed a surge in popularity due to the great volume in prediction markets related the 2024 US Presidential Election, with the highest volumes in prediction market history reached on modern prediction market platforms Polymarket and Kalshi. Multiple working papers are estimating the extent of price discovery through this market, such as in \textcite{ng_price_2025}, \textcite{cutting_are_2025}, and \textcite{jain_election_2025}. However, none are examining price discovery through Polymarket's prediction market series on the interest rate decisions of the Federal Open Market Committee (FOMC), and comparing this to the price discovery of traditional financial markets' futures data.

Prediction markets have been studied for three decades across political, corporate, and sports contexts, with a central topic of whether market prices can be read as forecasts of event probabilities.
\textcite{wolfers_prediction_2004} synthesise early evidence from the Iowa Electronic Markets (IEM), sports-focused prediction markets, and commercial prediction markets.
In \textcite{wolfers_interpreting_2006} they go on to analytically derivative that, under risk‐neutral pricing, contract prices can be interpreted as robust estimates of average beliefs about probabilities.
\textcite{arrow_promise_2008} agree with this perspective in their article published in the Science journal, and call for government regulation to harness potential scientific benefits in small cap prediction markets.
% A large body of work further evaluates the efficiency and informational content of prediction market prices.
\textcite{berg_prediction_2008} analyse US presidential election histories between 1988 and 2004 hosted by the IEM and find that prices are consistently well-set and estimates viewed as probabilities are competitive with standard, traditional political forecasting benchmarks.
% In a similar vein, evidence from historical political betting venues studied by
% Rhode and Strumpf (2008)
% is consistent with these findings and shows that even thin markets can aggregate information over election cycles, although liquidity and participation vary across eras.
\textcite{cowgill_corporate_2015} take a different approach in studying markets internal to corporations, such as at Google, Ford, and a third, anonymous firm, finding that prices track subsequent realisations fairly well and even outperform expert estimates in mean-squared error metrics and attribute this success to market mechanisms.

 % Beyond human polling, Laskey, Hanson, and Twardy (2015) motivate combinatorial prediction markets as a principled way to fuse signals from distributed experts and algorithmic models; they formalize how dependencies among events can be encoded and priced so that the market directly aggregates structured knowledge rather than a single headline probability. 

Another, still political but economics-focused strand of literature, concerns itself with using prediction markets' probabilities as proxies of public and investor sentiment. This literature examines potential information from prediction markets when combined alongside traditional financial markets.
\textcite{leigh_what_2003} present a
case study on the Iraq war which uses prediction market prices as conflict probabilities and compares them to price movements in oil, equities, and other asset classes.
\textcite{snowberg_partisan_2007} use prediction markets to infer who the public thought was going to win the 2000 and 2004 US presidential elections. The events' price-implied probabilities of a Republican or Democratic potential winner are paired with futures data, which reflect similar expected economic impacts.

A third, more recent strand of literature focuses on microstructural and arbitrage considerations regarding prediction markets.
\textcite{saguillo_unravelling_2025} conduct a large-scale arbitrage study on Polymarket, using on-chain order-book data together with LLM-powered arbitrage detection to analyse systematic price inconsistencies and realised arbitrage profits.
\textcite{ng_price_2025} consider cross-platform (Polymarket and Kalshi) arbitrage opportunities and price discovery by deep-pocketed market participants during the last days of the largest prediction market event in history as of writing this thesis, the 2024 US Presidential election.

Across these literatures, methods vary with the question. In research related to efficiency and accuracy, researchers rely on calibration curves scoring rules. In market-microstructure and mispricing, the latest Polymarket research uses algorithmic detection. Where prediction markets are combined with traditional assets, event studies and factor-style decompositions are prevalent.

This thesis relates to these three main strands of literature, addressing a gap where all of them meet: a one-to-one, high-frequency comparison of modern prediction market probabilities and prices in a directly corresponding, liquid futures market to address the question of market efficiency.
Much of the classic efficiency and accuracy evidence aggregates average forecast performance. 
Studies that connect prediction markets to financial assets typically use short cross-sectional event windows.
From a data perspective, high-volume prediction markets exist on Polymarket for every Federal Open Market Committee (FOMC) interest rate decision, yet no research has been done on this topic.

% which compares these markets' expectations with risk-neutral probabilities of the interest rate decisions implied by futures markets.
Minute-by-minute price discovery is analysed by continuously matching prediction market contracts to derivatives traded on the Chicago Mercantile Exchange that price the same event. 
By converting granular Polymarket and 30-Day Federal Funds rate futures into risk-neutral price-implied probabilities and testing for short-horizon price discovery, the thesis provides an assessment of whether prediction markets lead traditional ones in incorporating news about FOMC outcomes.
The econometric methodology aligns with standard practice in financial time-series analysis, drawing on the VECM framework, Johansen methodology, and Granger as well as instantaneous causality tests to single out, examine, and compare short-run price-adjustment.
