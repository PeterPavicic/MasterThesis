\section{Results} \label{sec:results}

This section reports the main empirical findings from the pooled the event-level analysis. It highlights patterns in the directional Granger causality tests, the instantaneous causality tests, and the rolling “which-lags-which” tests. Detailed robustness check tables can be found in Appendix
\ref{appendix:robustness_check}. 

% 1 minute
\subsection{Granger Causality}

% PM --> ZQ eigen
\begin{table}[H] 
\centering
\captionsetup{width=0.60\textwidth}
\caption{Blockwise Granger causality test PM $\longrightarrow$ ZQ}
\label{table:gc_1min}
\begin{tblr}[
]
{
colspec={Q[]Q[]Q[]Q[]Q[]Q[]},
column{1-5}={}{halign=l,},
}
\toprule
Name & F-statistic & $\text{df}_1$ & $\text{df}_2$ & p-value \\ \midrule
Pooled & 8.7708 & 96 & 3910648 & $<$2e-16 *** \\
\midrule
2023 February & 0.5908 & 54 & 149428 & 0.9928  \\
2023 March & 6.2489 & 60 & 238410 & $<$2e-16 *** \\
2023 May & 3.4130 & 12 & 172136 & $<$2e-16 *** \\
2023 June & 1.4808 & 42 & 213225 & 0.0230 * \\
2023 July & 0.7328 & 12 & 131310 & 0.7205  \\
2023 September & 0.5797 & 11 & 112926 & 0.8471  \\
2023 November & 1.2375 & 9 & 82726 & 0.2664  \\
2023 December & 0.4273 & 16 & 106294 & 0.9763  \\
2024 January & 0.5676 & 96 & 213844 & 0.9998  \\
2024 March & 1.6064 & 30 & 245890 & 0.0190 * \\
2024 May & 2.3942 & 15 & 114856 & 0.0018 ** \\
2024 June & 1.9469 & 42 & 171556 & 0.0002 *** \\
2024 July & 1.0100 & 12 & 200788 & 0.4361  \\
2024 September & 4.5131 & 48 & 278940 & $<$2e-16 *** \\
2024 November & 3.8492 & 345 & 789528 & $<$2e-16 *** \\
2024 December & 0.7360 & 240 & 966889 & 0.9992  \\
2025 January & 0.5480 & 64 & 508908 & 0.9988  \\
2025 March & 0.4277 & 138 & 467355 & 1  \\
2025 May & 3.4771 & 27 & 426084 & $<$2e-16 *** \\
2025 June & 0.8399 & 120 & 674600 & 0.8979  \\
2025 July & 1.1158 & 138 & 683220 & 0.1669  \\
\bottomrule
\end{tblr}
\caption*{The table shows the statistical output for the blockwise Granger causality tests in the direction Polymarket $\longrightarrow$ ZQ Futures}
\end{table}

% ZQ --> PM eigen
\begin{table}[H] 
\centering
\captionsetup{width=0.60\textwidth}
\caption{Blockwise Granger causality test ZQ $\longrightarrow$ PM}
\label{table:gc_1min2}
\begin{tblr}[
]
{
colspec={Q[]Q[]Q[]Q[]Q[]Q[]},
column{1-5}={}{halign=l,},
}
\toprule
Name & F-statistic & $\text{df}_1$ & $\text{df}_2$ & p-value \\ \midrule
Pooled & 7.1393 & 96 & 3910648 & $<$2e-16 *** \\
\midrule
2023 February & 1.0932 & 54 & 149428 & 0.2967  \\
2023 March & 5.0076 & 60 & 238410 & $<$2e-16 *** \\
2023 May & 4.0590 & 12 & 172140 & $<$2e-16 *** \\
2023 June & 1.0953 & 42 & 213220 & 0.3100  \\
2023 July & 0.6810 & 12 & 131310 & 0.7715  \\
2023 September & 0.1070 & 11 & 112926 & 0.9999  \\
2023 November & 5.9940 & 9 & 82726 & $<$2e-16 *** \\
2023 December & 0.5485 & 16 & 106294 & 0.9223  \\
2024 January & 0.8695 & 96 & 213844 & 0.8154  \\
2024 March & 0.0319 & 30 & 245890 & 1  \\
2024 May & 1.9907 & 15 & 114856 & 0.0124 * \\
2024 June & 2.2777 & 42 & 171556 & $<$2e-16 *** \\
2024 July & 0.6617 & 12 & 200788 & 0.7898  \\
2024 September & 26.1722 & 48 & 278940 & $<$2e-16 *** \\
2024 November & 8.8129 & 345 & 789528 & $<$2e-16 *** \\
2024 December & 0.7939 & 240 & 966889 & 0.9919  \\
2025 January & 1.7429 & 64 & 508908 & 0.0002 *** \\
2025 March & 1.7862 & 138 & 467355 & $<$2e-16 *** \\
2025 May & 11.2943 & 27 & 426084 & $<$2e-16 *** \\
2025 June & 0.8579 & 120 & 674600 & 0.8673  \\
2025 July & 2.1178 & 138 & 683220 & $<$2e-16 *** \\
\bottomrule
\end{tblr}
\caption*{The table shows the statistical output for the blockwise Granger causality tests in the direction ZQ Futures $\longrightarrow$ Polymarket}
\end{table}

Tables \ref{table:gc_1min} and \ref{table:gc_1min2} show the empirical results for Granger causality tests in each direction, for the event-level, as well as for the pooled dataset.

In the pooled sample, statistically strong predictability is found in both directions. The null that Polymarket (PM) does not Granger-cause ZQ is rejected at conventional significance levels $(F \approx 8.77$, $p < 2\times10^{-16}$), and the reverse null that ZQ does not Granger-cause PM is likewise rejected ($F \approx 7.14$, $p < 2\times10^{-16}$). This indicates that both series contain significant incremental predictive content for each other in this pooled dataset. 

At the level of individual FOMC meetings, the pattern is more heterogeneous. Several meetings show statistically significant PM $\longrightarrow$ ZQ predictability both early on in the dataset, in March, May, and June 2023.
For 2024, this is the case for 5 out of 8 markets, in March, May, June, September, and November 2024). For 2025, only the May data shows statistical significance.
Similarly, ZQ $\longrightarrow$ PM predictability is significant in a number of meetings, with differences compared to the other direction in June and November 2023, as well as March 2024, January, March, and May 2025 results.
Across meetings, highly significant rejections tend to occur in both directions, but there are also months with asymmetric rejections with Granger causality significant in only one direction. Overall, roughly half of the meetings display at least one direction of significance, with the precise direction varying by meeting.
The ZQ $\longrightarrow$ PM direction, aka the Polymarket lagging seems to be statistically significant more frequently, than the other way around. This is the case for 11 events, while in the PM $\longrightarrow$ ZQ direction this only happens 9 times.

Regarding robustness checks, the results agree for the pooled dataset across all checks.
For the event-level dataset, when excluding Mondays, as shown in Appendix \ref{appendix:no_monday}, the PM $\longrightarrow$ ZQ direction does not reject non-causality for the 2024 November event, while the ZQ $\longrightarrow$ PM direction rejects for the June 2023 and January 2024 months in addition to the main findings. 
Considering the robustness check for aggregating data at 5-minute fidelity in Appendix \ref{appendix:five_minute}, the 2024 May and June events no longer reject non-causality for both directions, for Polymarket-leads-ZQ this is also the case for May 2023. On the other hand, for ZQ-leads-Polymarket robustness, the 2023 February and 2025 June markets reject non-causality.
The trace statistics agree with the main maximum-eigenvalue statistic based findings across all datasets and outputs (as shown in Appendix \ref{appendix:trace}).


% 1 minute
\subsection{Instantaneous Causality}

% symmetric
\begin{table}[H] 
\centering
\captionsetup{width=0.60\textwidth}
\caption{Blockwise Instantaneous causality test}
\label{table:ic}
\begin{tblr}[
]
{
colspec={Q[]Q[]Q[]Q[]Q[]},
column{1-4}={}{halign=l,},
}
\toprule
Name & $\chi^2$ & $\text{df}_1$ & p-value \\ \midrule
Pooled & 1466.0978 & 4 & $<$2e-16 *** \\
\midrule
2023 February & 0.1067 & 3 & 0.9910  \\
2023 March & 22.7964 & 6 & 0.0009 *** \\
2023 May & 1.7779 & 3 & 0.6198  \\
2023 June & 7.1283 & 6 & 0.3091  \\
2023 July & 0.7415 & 2 & 0.6902  \\
2023 September & 0.3182 & 1 & 0.5727  \\
2023 November & 3.7030 & 1 & 0.0543 . \\
2023 December & 0.0655 & 1 & 0.7981  \\
2024 January & 2.0396 & 4 & 0.7285  \\
2024 March & 2185.1144 & 6 & $<$2e-16 *** \\
2024 May & 4.3906 & 3 & 0.2223  \\
2024 June & 1.6870 & 3 & 0.6398  \\
2024 July & 5.3890 & 3 & 0.1454  \\
2024 September & 80.6587 & 6 & $<$2e-16 *** \\
2024 November & 32.5638 & 15 & 0.0054 ** \\
2024 December & 7.1102 & 10 & 0.7150  \\
2025 January & 11.7129 & 8 & 0.1645  \\
2025 March & 14.6879 & 6 & 0.0228 * \\
2025 May & 14.8822 & 3 & 0.0019 ** \\
2025 June & 8.0940 & 6 & 0.2313  \\
2025 July & 27.1094 & 6 & 0.0001 *** \\
\bottomrule
\end{tblr}
\caption*{The table shows the statistical output for the blockwise instantaneous causality tests. The tests measures contemporaneous comovement, and would therefore be \enquote{symmetric} in both directions.}
\end{table}


Since instantaneous causality measures contemporaneous co-movement in the blocks, it is symmetric.
The results shown in Table \ref{table:ic} show for the
pooled test that it is decisively significant with ($\chi ^2 \approx 1466$, df $= 4, p < 2 \times 10^{-16})$.
At the event level, instantaneous causality is significant for a third of all meetings, that is, 7 out of 21, while most meetings do not reject at conventional thresholds.

Regarding robustness checks, the November 2024 event does not reject non-causality when excluding data from Mondays, while aggregating at the 5-minute instead of 1-minute intervals does not reject for March 2024 but does reject for May 2023 and May 2024 in addition to these main results.
The pooled dataset's robustness checks agree with the main findings fully.

Trace statistics once again fully agree with the main methodology's findings, and the robustness checks' findings.

\subsection{Which lags which}

% which lags which eigen, GC
\begin{table}[H] 
\centering
\caption{Blockwise Granger causality which-lags-which findings}
\captionsetup{width=0.60\textwidth}
\label{table:wlw}
\begin{tblr}[
]
{
colspec={Q[]Q[]Q[]},
column{1-3}={}{halign=l,},
}
\toprule
Name & PM $\longrightarrow$ ZQ & ZQ $\longrightarrow$ PM \\ \midrule
Pooled & 0.4314 & 0.4575 \\ 
\midrule
2023 February & 0.4706 & 0.6118 \\
2023 March & 0.5991 & 0.4340 \\
2023 May & 0.3667 & 0.2444 \\
2023 June & 0.3985 & 0.3443 \\
2023 July & 0.0109 & 0.0929 \\
2023 September & 0.2579 & 0.3415 \\
2023 November & 0.1783 & 0.2868 \\
2023 December & 0.0438 & 0.0438 \\
2024 January & 0.2971 & 0.4377 \\
2024 March & 0.1418 & 0.0982 \\
2024 May & 0.2936 & 0.2752 \\
2024 June & 0.4320 & 0.3850 \\
2024 July & 0.1759 & 0.2556 \\
2024 September & 0.1753 & 0.3299 \\
2024 November & 0.3925 & 0.4677 \\
2024 December & 0.3333 & 0 \\
2025 January & 0.1011 & 0.4494 \\
2025 March & 0.2802 & 0.4258 \\
2025 May & 0.3164 & 0.4369 \\
2025 June & 0.8033 & 0.8525 \\
2025 July & 0.2642 & 0.3396 \\
\bottomrule
\end{tblr}
\caption*{The table shows results for rolling window blockwise Granger causality tests. Each entry in the table shows the ratio of 1-day windows rejecting Granger non-causality in the direction determined by the column names to all windows of such length.}
\end{table}

The rolling \enquote{which-lags-which} analysis shown in Table \ref{table:wlw} summarises how frequently each market leads the other across moving windows. For the pooled analysis, the reported shares are of similar magnitude for both directions (approximately 0.43 for PM $\longrightarrow$ ZQ and 0.46 for ZQ $\longrightarrow$ PM). At the meeting level the shares vary widely compared to one-another.
When compared in pairs, some meetings show relatively balanced values near the pooled figures, some tilt toward one direction (such as several 2024 and 2025 meetings, which tilt toward ZQ predicting PM), and a few meetings exhibit pronounced asymmetry (for example December 2024, where ZQ is not at all significant in Granger causing PM data). 
For 13 out of 21 meetings, the ZQ data predicts Polymarket data for a larger share of the total data timeframe. The other way around this holds for 7 meetings out of 21.
The overall heterogeneity across meetings suggests that lead-lag patterns heavily depend on the specific meeting. 

Overall, the results seem to indicate that for the pooled data, predictive relationships are heavily present in both directions, while the meeting-level outcomes vary quite a bit across meetings.

In the robustness checks, the trace statistics agree with the findings of the maximum-eigenvalue statistic.

