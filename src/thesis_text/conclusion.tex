\section{Conclusion}


% Against the backdrop of the classic prediction-market literature, the contribution lies in identifying the micro-timing of who moves first when two markets price the same FOMC event. Prior work emphasises calibration and average forecasting accuracy; the present design adds a high-frequency, one-to-one comparison against a liquid benchmark and shows that modern prediction markets do not merely echo professional venues, but sometimes precede them. That extends the literature from static accuracy into dynamic price discovery and clarifies that Polymarket’s informational role is meaningful yet contingent on the informational environment and trading mechanics of a given meeting. 

This thesis investigates price discovery and information aggregation behaviour observed on the contemporary prediction market Polymarket.
It does so by examining a series of prediction markets on the interest rate decision of the Federal Reserve's Federal Open Market Committe,
interpreting Polymarket's prices as risk-neutral probabilities, and comparing them to risk-neutral probabilities extracted from 30-Day Federal Funds futures through the CME's FedWatch tool.
A VAR/VECM framework is deployed with directional Granger and instantaneous causality testing on both pooled and meeting-by-meeting data to analyse whether prediction markets or traditional financial markets move first when news regarding upcoming changes to interest rate policy arrive.

The research shows both sources are rather efficient when it comes to price discovery.
In pooled analyses, predictive power is highly significant in both directions, and contemporaneous dependence is strong, indicating that information often arrives and is absorbed nearly simultaneously.
At the meeting level, the evidence is mixed. Findings from some FOMC meetings show Polymarket leading price discovery, others show futures data leading.
The overall picture indicates equal importance rather than complete superiority of either source.
Robustness checks show slightly more reliable futures data, while findings for Polymarket are also affirmed as statistically valid and robust.

Limitations of this work are important for considering future research on this topic. The scope is limited to a single prediction market platform with a shared topic. Further research are to be done in this data-rich field extending to further platforms, event topics, and applications. The analysis conducted in this thesis offers a starting point.
