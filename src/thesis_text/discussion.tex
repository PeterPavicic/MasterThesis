\section{Discussion}

The evidence presented in this thesis shows that both Polymarket and ZQ futures perform meaningful price discovery.
In the pooled sample that rolls all FOMC meetings, always examining the nearest one, blockwise Granger causality tests reject in both directions with very small p-values, showing that each set of data contains incremental predictive content for the other beyond its own lags.
This can be interpreted as a decisive finding that neither market is a passive follower on average across meetings. It is however, important to mention the limitation of the data-pooling method which was deployed in the creation of this dataset. Since the nearest event is rolled throughout the data, right after each FOMC decision is announced, a new dataset is rolled into. As the plotted per-meeting implied probabilities in Appendix \ref{appendix:all_timeseries} show, this means that the implied probabilities jump from close to 0/1 (depending on whether an interest rate decision realises at the meeting or not) to a more ambiguous, lower probability, since the next decision is yet still at least 1 month away. Instantaneous causality results further support this finding, showing highly statistically significant contemporaneous shocks, which cannot be explained by lagged variables in either timeseries.
Controlling for this effect, while beyond the scope of this thesis, could provide valuable information regarding the robustness of findings for this dataset.

Event-level tests support these findings. For several meetings, Polymarket leads subsequent changes in ZQ’s implied probabilities, and vice versa, while the meetings where neither leads the other is in minority (8 out of 21). This becomes even more pronounced in more recent markets, since June 2024, which have enjoyed higher volumes compared to before.
Still, the statistical significance of Granger causality tests is not fully consistent in either direction throughout time, which
indicates that the lead-lag relationship examined in the thesis is more meeting-dependent than the inference on the pooled seems to indicate. 

Rolling \enquote{which-lags-which} shares count rejections of non-causality, and summarise how frequently each market leads the other within moving one-day windows. In the pooled data, those shares are roughly balanced (about 0.43 for PM $\longrightarrow$ ZQ and 0.46 for ZQ $\longrightarrow$ PM), while at the meeting level they disperse widely, including cases of extreme asymmetry such as December 2024 where ZQ never leads Polymarket data statistically significantly. The magnitudes of the findings for this analysis also vary widely, further supporting the heterogeneity in results found previously.

Instantaneous causality is large in the aggregate and present in about a third of meetings.
As mentioned before, the pooled instantaneous test strongly rejects the null of no contemporaneous covariance, and several meetings show the same, implying that in these cases, implied probabilities tend to adjust essentially simultaneously across both sources rather than participants of one of the two markets implementing information before the other.

From a microstructural perspective, it is important to consider traders' positions. FOMC meeting information is concentrated and released in a well-established manner (at 14:00 New York time, on the 2nd meeting day). Statement texts such as the Summary of Economic Projections (SEP) are standardised, enabling market participants' use of algorithmic tools, benefitting primarily the institutional investor-heavy futures markets. However, when information is diffuse and accumulates outside exchange hours, Polymarket’s 24/7 avilability lets its traders re-price earlier.
The data show both cases: March 2024 can be interpreted as a clear example of Polymarket informationally outperforming, while January and July 2025 are instance of the opposite. Again, neither is consistent. 

While results are generally robust for all findings, it is interesting to remark that in cases of disagreements with robustness checks, the futures market data seemed to be more conservative towards rejection of non-causality vis-a-vis Polymarket data at the selected 1-minute frequency than the other way around. Combined with the higher number of statistically significant cases of causality, and higher count of significant windows in the which-lags-which analysis, this indicates that the futures markets are overall slightly faster at information aggregation and price discovery compared to prediction markets, as exemplified by Polymarket. 

Main takeaways can be summarised into three main points. First, there is no single, consistent \enquote{informational leader} across the sample. This is evident in the event-level Granger causality tests and in the rolling which-lags-which tests, albeit with a slight advantage to the futures markets.
Second, the evidence of predictability in both directions shows that both types of markets contribute to price discovery on average, consistent with semi-strong market efficiency in which common shocks often are absorbed at the same time, and lagged price-adjustment in one market is often quickly followed in the other.
Third, these inferences are robust to removing Mondays, to re-sampling at five minutes, and to cointegration rank selection methodology.
This reduces concerns that weekend effects, market microstructural noise, or cointegration diagnostics irregularities drive the findings.

As for the research question: How much price discovery takes place on Polymarket and whether it is efficient? The evidence indicates that Polymarket's price discovery mechanism is not trivial.
It incorporates new information as quickly as the futures market in many instances and earlier in some, while in others it swiftly learns from them.
This shows that modern prediction markets can be efficient and timely aggregators of information, keeping up with, and occasionally leading traditional financial markets, albeit inconsistently, pointing at some remaining structural inefficiencies.
